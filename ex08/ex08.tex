\documentclass[a4paper]{article}
\usepackage[english]{babel}
\usepackage[utf8]{inputenc}

% mathermatics
\usepackage{amssymb} % useful math symbols
\usepackage{mathrsfs,amsmath} % more useful math (mathrsfs for Fourier F)

% graphics
\usepackage{graphicx}
\usepackage{float}    % for more accurate graphics placement
\usepackage{fancyhdr} % for top header

% references
\usepackage{hyperref} % needed by cleveref, also provides clickable links
\usepackage{cleveref} % needed by autonum
\usepackage{autonum} % only add numbers to referenced equations

% formatting
\usepackage{enumitem} % provides easy change of labels in enumerate environment
\usepackage[top=3cm, bottom=4cm, width=17cm]{geometry} % for smaller page margins

% colors
\usepackage{xcolor}

% coding
% 
% uncomment this after you have completed the required installation
% see https://github.com/gpoore/minted for info

\usepackage{minted} 
\definecolor{codeBgColor}{RGB}{240,240,240}




% very simple alias, \ex{} becomes the same as \subsubsection*{}
% TIP: remove the * in the line below if you want it numbered
\newcommand{\ex}[1]{\subsubsection*{#1}}




%Begining of the document
\begin{document}

\pagestyle{fancy} % use pagestyle with simple header (from fancyhdr)

%\pagenumbering{gobble} % uncomment to remove pagenumbering (in case of single page document)
\fancyhead[L]{TMA4135 Matematikk 4D}
\fancyhead[C]{\textbf{Exercise 8}}
\fancyhead[R]{Otto Lote (748704)}
\fancyfoot{}

\ex{1}

\begin{enumerate}[label=\alph*)]
    \item
        \begin{align}
            y'' + 4y &= e^t \\
            \intertext{We laplace transform both sides of the equation and get}
            s^2\hat y &- sy(0) - y'(0) + 4 \hat y = \frac{1}{s-1}, \quad y(0) = y'(0) = 0\\
            \hat y &= \frac{1}{(s-1)(s^2 + 4)}\\
            \hat y &= \frac{0.2}{s-1} + \frac{-0.1 + 0.05i}{s-2i} + \frac{-0.1
                - 0.05i}{s+2i} \\
            \intertext{taking the inverse laplace transform yields}
            y &= 0.2e^t + ( -0.1 + 0.05i)e^{2it} + ( -0.1 - 0.05i)e^{-2it} \\
            y &= 0.2e^t -0.2\frac{e^{2it}+e^{-2it}}{2} - 0.1\frac{e^{2it} -
                e^{-2it}}{2i} \\
            y &= 0.2e^t - 0.2\cos(2t) - 0.1\sin(2t) \\
            y &= \frac{1}{10}\big(2e^t - 2\cos(2t) - 1\sin(2t)\big) \\
        \end{align}

    \item
        \begin{align}
            y(t) &- \int_0^t{y(\tau)(t-\tau) d\tau} = 2 - \frac{1}{2}t^2 \\
        \end{align}
        ??? - Missing second function for convolution?
\end{enumerate}


\newpage
\ex{2}

\begin{align}
    ty'' &- ty' +y = 1, \quad y(0) = 1, \quad y'(0) = 2 \\
    \intertext{Taking the laplace transform of both sides gives us}
    -(s^2\hat y &- sy(0) - y'(0)) -(s\hat y - y(0)) + \hat y = \frac{1}{s} \\
    -s^2\hat y &+ s + 2 - s\hat y + 1 + \hat y = \frac{1}{s} \\
    \hat y(-s^2 & - s + 1) = \frac{1}{s} - 3 - s \\
    \hat y &= -\frac{1}{s(s^2 + s - 1)} + \frac{3}{s^2 + s - 1} + \frac{s}{s^2 +s -1} \\
    \intertext{\(s^2 + s -1\) can be written as \((s -r_1)(s - r_2)\) where 
        \(r_1 = \frac{-1 + \sqrt{5}}{2}, \quad r_2 = \frac{-1 - \sqrt{5}}{2}\)
        (and \((r_1 -r_2) = \sqrt{5}\))}
    \hat y &= -\frac{1}{s(s-r_1)(s-r_2)} + \frac{3}{(s-r_1)(s-r_2)} +
        \frac{s}{(s-r_1)(s-r_2)} \\
    \intertext{We know that}
    \frac{1}{(s-r_1)(s-r_2)} &= \frac{1}{(r_1-r_2)(s-r_1)} + \frac{1}{(r_2-r_1)(s-r_2)}\\
    &= \frac{1}{\sqrt{5}(s-r_1)} -
        \frac{1}{\sqrt{5}(s-r_2)}\\
    \hat y &= \frac{1}{\sqrt{5}} \Bigg(\frac{1}{s(s-r_1)} - \frac{1}{s(s-r_2)}
    \intertext{Taking the inverse Laplace transform yields}
    y &= \frac{1}{\sqrt{5}} \Bigg(\frac{1}{r_1}e^{r_1t} - \frac{1}{r_2}e^{r_2t}
        + 3e^{r_1t} - 3e^{r_2t} + r_1e^{r_1t} - r_2e^{r_2t}\Bigg) \\
    &= \frac{1}{\sqrt{5}} e^{r_1t}\bigg( \frac{1}{r_1} + 3 + r_1 \bigg) 
        - e^{r_2t} \bigg( \frac{1}{r_2} + 3 + r_2 \bigg) \\
    &= \frac{1}{\sqrt{5}} e^{\frac{-1 + \sqrt{5}}{2}t}\bigg( \frac{1}{\frac{-1
        + \sqrt{5}}{2}} + 3 + \frac{-1 + \sqrt{5}}{2} \bigg) - e^{\frac{-1 -
        \sqrt{5}}{2}t} \bigg( \frac{1}{\frac{-1 - \sqrt{5}}{2}} + 3 + \frac{-1
        - \sqrt{5}}{2} \bigg) \\
    &= \frac{1}{\sqrt{5}} e^{\frac{-1 + \sqrt{5}}{2}t}\bigg( \frac{1}{\frac{-1
        + \sqrt{5}}{2}} + 3 + \frac{-1 + \sqrt{5}}{2} \bigg) - e^{\frac{-1 -
        \sqrt{5}}{2}t} \bigg( \frac{1}{\frac{-1 - \sqrt{5}}{2}} + 3 + \frac{-1
        - \sqrt{5}}{2} \bigg) \\
    &= 2.3416e^{0.6180t} - 0.3416e^{-1.6180t} \\
\end{align}

\ex{3}

\begin{enumerate}[label=\alph*)]
    \item When \(s\) is a fixed point of \(g\) then \(g(s) = s\) which means that
                    \(g^{-1}(s) = s\) by definition of the inverse
    \item 
        \begin{align}
            \intertext{Since g maps from \(x\) to \(y\) we can set}
            y &= g(x)  \label{eq:3-ydef}\\
            \intertext{and its inverse \(g^{-1}\) maps from \(y\) to \(x\)}
            g^{-1}(y) &= x \\
            \intertext{Differentiating both sides with respects to y gives us}
            \frac{dg^{-1}}{dy} &= \frac{dx}{dy} \label{eq:3-1}\\
            \intertext{Insert (\ref{eq:3-ydef}) into (\ref{eq:3-1}) yields}
            \frac{dg^{-1}}{dy} &= \frac{dx}{dg(x)} = \frac{1}{g'(x)} \\
        \end{align}

    \item {
        \begin{align}
            \arccos(x) &= x \\
            \intertext{We know that the iteration process \(x_{n+1} = g(x_n)\)
                converges towards a solution \(s = g(s)\) if \(|g'(x)| < 1\)}
            \intertext{From 3a) we know that a fixed point for a function \(g\)
                is also a fixed point in its inverse \(g^{-1}\)}
            \intertext{Since \((g^{-1})'(x) = \frac{d}{dx} \cos(x) = -\sin(x)\)
                has an absolute value less than one on the interval
                \((\frac{-\pi}{2},\frac{\pi}{2})\) which means the iteration
                process}
            x_{n+1} &= \cos(x_n), \quad x \in (\frac{-\pi}{2},\frac{\pi}{2}) \\
            \intertext{will converge towards a fixed point solution \(s\)}
        \end{align}

        The iteration can be implemented very simply in python

        \begin{minted}
            [
                linenos,                % line numbers
                bgcolor=codeBgColor     % background color, remove for none
            ]
            {python} 
from math import cos

s = 0

# 100000 iterations is overkill and way more than sufficient
for i in range(100000):
    s = cos(s)

# print with two significant digits
print('%.2f' % s)

        \end{minted}

        Output:

        \begin{minted}
            [
                bgcolor=codeBgColor     % background color, remove for none
            ]
            {bash} 
$ python 3c.py                                               
0.74
        \end{minted}
    }

    \item 
        \begin{align}
            \intertext{The distance from the solution during an iteration is given by}
            |x_n - s| &= |g(x_{n-1}) - g(s)| = |g'(x)||x_{n-1} - s| \leq
                K|x_{n-1} - s|\\
            \intertext{Or in our case}
            |x_n - s| &= |g^{-1}(x_{n-1}) - g^{-1}(s)| = |(g^{-1})'(x)||x_{n-1}
                - s| \leq K|x_{n-1} -s|\\
            \intertext{Where \(K\) is the maximum value of \(|(g^{-1})'(x)|\)
                on the interval \([x_0,s]\)}
            |(g^{-1})'(x)| &= \sin(x), \quad x \in [x_0,s] \\
            \intertext{For \(x_0 = 0.5\) and \(s < 0.74\) 
                \begingroup
                    \fontsize{6pt}{8pt}\selectfont
                    (s = 0.7390851332151607)
                \endgroup
                \(|(g^{-1})'(x)|\) is increasing as \(x\)
                increases, so we can set \(K = 0.74\)}
            \intertext{For \(n\) iterations we have}
            |x_n - s| &\leq K^n|x_0 - s| = 0.74^n|0.5-0.74| = 0.24 \cdot 0.74^n\\
            \intertext{To find the necessary steps for the error \(|x_n - s| < 10^{-15}\) we solve the inequality}
            0.24 \cdot 0.74^n < 10^{-15} \\
            \log0.24 + n\log0.74 < -15 \\
            n > \frac{-15 - \log0.24}{\log0.74} \approx 109.97\\
            \intertext{This means that 110 iterations is sufficient for an
                error of less than \(10^{-15}\)}
        \end{align}
\end{enumerate}

\newpage
\ex{4}

\begin{align}
    x &= \sqrt[3]{7} \\
    \intertext{we have the function}
    f(x) &= x^3 - 7 = 0 \\
    \intertext{and its derivative}
    f'(x) &= 3x^2 \\
    \intertext{We have Newton's method defined as}
    x_{n+1} = x_n - \frac{f(x_n)}{f'(x_n)} = x_n - \frac{x^3 - 7}{3x^2}\\
\end{align}

We can implement this in python in the following way

\begin{minted}
    [
        linenos,                % line numbers
        bgcolor=codeBgColor     % background color, remove for none
    ]
    {python} 
def f(x):
    return x**3 - 7

def f_d(x):
    return 3*x**2

def newton(x_n):
    return x_n - f(x_n)/f_d(x_n)

# start iterating from s = 1
s = 1

s_prev = 0

# Run until s = s_prev
while s != s_prev: 
    s_prev = s
    s = newton(s)

print('%.3f' % s)
\end{minted}

Output:

\begin{minted}
    [
        bgcolor=codeBgColor     % background color, remove for none
    ]
    {bash} 
$ python 4.py                                               
1.913
\end{minted}

%% uncomment this if you need references. Edit the .bbl file with your references
%% and use "\cite{bibitem-label}" to cite
%\bibliography{template}

\end{document}

