\documentclass[a4paper]{article}
\usepackage[english]{babel}
\usepackage[utf8]{inputenc}

% mathermatics
\usepackage{amssymb} % useful math symbols
\usepackage{mathrsfs,amsmath} % more useful math (mathrsfs for Fourier F)

% graphics
\usepackage{graphicx}
\usepackage{float}    % for more accurate graphics placement
\usepackage{fancyhdr} % for top header

% references
\usepackage{hyperref} % needed by cleveref, also provides clickable links
\usepackage{cleveref} % needed by autonum
\usepackage{autonum} % only add numbers to referenced equations

% formatting
\usepackage{enumitem} % provides easy change of labels in enumerate environment
\usepackage[top=3cm, bottom=4cm, width=17cm]{geometry} % for smaller page margins

% colors
\usepackage{xcolor}

% coding
% 
% uncomment this after you have completed the required installation
% see https://github.com/gpoore/minted for info

%\usepackage{minted} 
%\definecolor{codeBgColor}{RGB}{240,240,240}




% very simple alias, \ex{} becomes the same as \subsubsection*{}
% TIP: remove the * in the line below if you want it numbered
\newcommand{\ex}[1]{\subsubsection*{#1}}




%Begining of the document
\begin{document}

\pagestyle{fancy} % use pagestyle with simple header (from fancyhdr)

%\pagenumbering{gobble} % uncomment to remove pagenumbering (in case of single page document)
\fancyhead[L]{TMA4135 Matematikk 4D}
\fancyhead[C]{\textbf{Exercise 8}}
\fancyhead[R]{Otto Lote (748704)}

\ex{1}

\begin{enumerate}[label=\alph*)]
    \item
        \begin{align}
            y'' + 4y &= e^t \\
            \intertext{We laplace transform both sides of the equation and get}
            s^2\hat y &- sy(0) - y'(0) + 4 \hat y = \frac{1}{s-1}, \quad y(0) = y'(0) = 0\\
            \hat y &= \frac{1}{(s-1)(s^2 + 4)}\\
            \hat y &= \frac{0.2}{s-1} + \frac{-0.1 + 0.05i}{s-2i} + \frac{-0.1
                - 0.05i}{s+2i} \\
            \intertext{taking the inverse laplace transform yields}
            y &= 0.2e^t + ( -0.1 + 0.05i)e^{2it} + ( -0.1 - 0.05i)e^{-2it} \\
            y &= 0.2e^t -0.2\frac{e^{2it}+e^{-2it}}{2} - 0.1\frac{e^{2it} -
                e^{-2it}}{2i} \\
            y &= 0.2e^t - 0.2\cos(2t) - 0.1\sin(2t) \\
            y &= \frac{1}{10}\big(2e^t - 2\cos(2t) - 1\sin(2t)\big) \\
        \end{align}

    \item
        \begin{align}
            y(t) &- \int_0^t{y(\tau)(t-\tau) d\tau} = 2 - \frac{1}{2}t^2 \\
        \end{align}
        ??? - Missing second function?
\end{enumerate}

\ex{2}

\begin{align}
    ty'' &- ty' +y = 1, \quad y(0) = 1, \quad y'(0) = 2 \\
    \intertext{Taking the laplace transform of both sides gives us}
    -(s^2\hat y &- sy(0) - y'(0)) -(s\hat y - y(0)) + \hat y = \frac{1}{s} \\
    -s^2\hat y &+ s + 2 - s\hat y + 1 + \hat y = \frac{1}{s} \\
    \hat y(-s^2 & - s + 1) = \frac{1}{s} - 3 - s \\
    \hat y &= -\frac{1}{s(s^2 + s - 1)} + \frac{3}{s^2 + s - 1} + \frac{s}{s^2 +s -1} \\
    \intertext{\(s^2 + s -1\) can be written as \((s -r_1)(s - r_2)\) where 
        \(r_1 = \frac{-1 + \sqrt{5}}{2}, \quad r_2 = \frac{-1 - \sqrt{5}}{2}\)
        (and \((r_1 -r_2) = \sqrt{5}\))}
    \hat y &= -\frac{1}{s(s-r_1)(s-r_2)} + \frac{3}{(s-r_1)(s-r_2)} +
        \frac{s}{(s-r_1)(s-r_2)} \\
    \intertext{We know that}
    \frac{1}{(s-r_1)(s-r_2)} &= \frac{1}{(r_1-r_2)(s-r_1)} + \frac{1}{(r_2-r_1)(s-r_2)}\\
    &= \frac{1}{\sqrt{5}(s-r_1)} -
        \frac{1}{\sqrt{5}(s-r_2)}\\
    \hat y &= \frac{1}{\sqrt{5}} \Bigg(\frac{1}{s(s-r_1)} - \frac{1}{s(s-r_2)}
    \intertext{Taking the inverse Laplace transform yields}
    y &= \frac{1}{\sqrt{5}} \Bigg(\frac{1}{r_1}e^{r_1t} - \frac{1}{r_2}e^{r_2t}
        + 3e^{r_1t} - 3e^{r_2t} + r_1e^{r_1t} - r_2e^{r_2t}\Bigg) \\
    &= \frac{1}{\sqrt{5}} e^{r_1t}\bigg( \frac{1}{r_1} + 3 + r_1 \bigg) 
        - e^{r_2t} \bigg( \frac{1}{r_2} + 3 + r_2 \bigg) \\
    &= \frac{1}{\sqrt{5}} e^{\frac{-1 + \sqrt{5}}{2}t}\bigg( \frac{1}{\frac{-1
        + \sqrt{5}}{2}} + 3 + \frac{-1 + \sqrt{5}}{2} \bigg) - e^{\frac{-1 -
        \sqrt{5}}{2}t} \bigg( \frac{1}{\frac{-1 - \sqrt{5}}{2}} + 3 + \frac{-1
        - \sqrt{5}}{2} \bigg) \\
    &= \frac{1}{\sqrt{5}} e^{\frac{-1 + \sqrt{5}}{2}t}\bigg( \frac{1}{\frac{-1
        + \sqrt{5}}{2}} + 3 + \frac{-1 + \sqrt{5}}{2} \bigg) - e^{\frac{-1 -
        \sqrt{5}}{2}t} \bigg( \frac{1}{\frac{-1 - \sqrt{5}}{2}} + 3 + \frac{-1
        - \sqrt{5}}{2} \bigg) \\
    &= 2.3416e^{0.6180t} - 0.3416e^{-1.6180t} \\
\end{align}

\ex{3}

\begin{enumerate}[label=\alph*)]
    \item When \(s\) is a fixed point of \(g\) then \(g(s) = s\) which means that
                    \(g^{-1}(s) = s\) by definition of the inverse
    \item 
        \begin{align}
            \intertext{Since g maps from \(x\) to \(y\) we can set}
            y &= g(x)  \label{eq:3-ydef}\\
            \intertext{and its inverse \(g^{-1}\) maps from \(y\) to \(x\)}
            g^{-1}(y) &= x \\
            \intertext{Differentiating both sides with respects to y gives us}
            \frac{dg^{-1}}{dy} &= \frac{dx}{dy} \label{eq:3-1}\\
            \intertext{Insert (\ref{eq:3-ydef}) into (\ref{eq:3-1}) yields}
            \frac{dg^{-1}}{dy} &= \frac{dx}{dg(x)} = \frac{1}{g'(x)} \\
        \end{align}

    \item
        \begin{align}
            \arccos(x) &= x \\
            \intertext{We know that the iteration process \(x_{n+1} = g(x_n)\)
                converges towards a solution \(s = g(s)\) if \(|g'(x)| < 1\)}
            \intertext{Since }
        \end{align}
\end{enumerate}


\ex{4}

\begin{align}
\end{align}


%% uncomment this if you need references. Edit the .bbl file with your references
%% and use "\cite{bibitem-label}" to cite
%\bibliography{template}

\end{document}

