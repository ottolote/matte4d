\documentclass[a4paper]{article}
\usepackage[english]{babel}
\usepackage[utf8]{inputenc}

% mathermatics
\usepackage{amssymb} % useful math symbols
\usepackage{mathrsfs,amsmath} % more useful math (mathrsfs for Fourier F)

% graphics
\usepackage{graphicx}
\usepackage{float}    % for more accurate graphics placement
\usepackage{fancyhdr} % for top header

% references
\usepackage{hyperref} % needed by cleveref, also provides clickable links
\usepackage{cleveref} % needed by autonum
\usepackage{autonum} % only add numbers to referenced equations

% formatting
\usepackage{enumitem} % provides easy change of labels in enumerate environment
\usepackage[top=3cm, bottom=4cm, width=17cm]{geometry} % for smaller page margins

% colors
\usepackage{xcolor}

% coding
% 
% uncomment this after you have completed the required installation
% see https://github.com/gpoore/minted for info

%\usepackage{minted} 
%\definecolor{codeBgColor}{RGB}{240,240,240}




% very simple alias, \ex{} becomes the same as \subsubsection*{}
% TIP: remove the * in the line below if you want it numbered
\newcommand{\ex}[1]{\subsubsection*{#1}}




%Begining of the document
\begin{document}

\pagestyle{fancy} % use pagestyle with simple header (from fancyhdr)

%\pagenumbering{gobble} % uncomment to remove pagenumbering (in case of single page document)
\fancyhead[L]{TMA4135 Matematikk 4D}
\fancyhead[C]{\textbf{Exercise 6}}
\fancyhead[R]{Otto Lote (748704)}

\ex{1}

\begin{align}
    \intertext{We have}
    \frac{\partial^2u}{\partial x^2} + \frac{\partial^2u}{\partial y^2} &= 0
        ,\quad y>0, \, x \in \mathbb{R} \label{eq:1-PDE}\\
    \intertext{with boundary conditions}
    u(x,0) &= f(x) \\
    \lim_{x\to\infty} u(x,y) &= 0 \label{eq:1-bound-x-inf}\\
    \lim_{x\to-\infty} u(x,y) &= 0 \label{eq:1-bound-xinf}\\
    \lim_{y\to\infty} u(x,y) &= 0 \label{eq:1-bound-yinf}\\
\end{align}

\begin{enumerate}[label=\alph*)]
    \item
        \begin{align}
            \hat u(w,y) &= \mathscr{F}(u)(w,y) \\
            \intertext{Since the Fourier transform is an integral with respects
                to \(x\) we can move the partial derivative with respect to y
                in front of the integral, and therefore in front of the Fourier
                transform.}
            \mathscr{F}\Big(\frac{\partial^2u}{\partial y^2}\Big) &= 
                \frac{\partial^2}{\partial y^2}\mathscr{F}(u) = 
                \frac{\partial^2\hat u}{\partial y^2}\\
            \intertext{We know that}
            \mathscr{F}(u^{(n)}(x)) &= (iw)^n\hat u(w) \\
            \intertext{which also holds for partial derivatives}
            \mathscr{F}\Big( \frac{\partial^2u}{\partial x^2}\Big) &= -w^2\hat u \\
            \intertext{Taking the Fourier transform of the PDE in
                (\ref{eq:1-PDE}) gives us}
            \frac{\partial^2\hat u}{\partial^2y} - w^2\hat u &= 0
        \end{align}

    \item
        \begin{align}
            \intertext{If we write \(u = F(x)G(y)\) we can write the PDE as}
            F''G + FG'' &= 0 \\
            \intertext{Rearranging we get} 
            \frac{F''}{F} &= - \frac{G''}{G} = k \\
            \intertext{This is constant since \(F\) and \(G\) depend on
                different variables. We can now form two ODEs}
            F'' -kF &= 0 \\
            G'' +kG &= 0 \\
            \intertext{If we set positive \(k = \mu^2\) then \(G(y)\) has a
                general solution that doesn't converge towards 0 as \(y \to
                \infty\), so we set negative \(k = -p^2\) and get the general
                solution}
            G(y) &= G(0)\cosh(py) - G'(0) \frac{1}{p}\sinh(py) \\
%            &= \frac{1}{2} \Big( G(0)(e^{py} + e^{-py}) -
%                \frac{G'(0)}{p}(e^{py} - e^{-py}) \Big) \\
%            &= \frac{1}{2} \Big( G(0)e^{py} + G(0)e^{-py} -
%                \frac{G'(0)}{p}e^{py} + \frac{G'(0)}{p}e^{-py} \Big) \\
            &= \frac{1}{2} \Big( \big(G(0)-\frac{G'(0)}{p}\big)e^{py} + \big(G(0) +
                \frac{G'(0)}{p}\big)e^{-py} \Big) \\
        \end{align}
\end{enumerate}

\ex{2}

\begin{align}
    \mathscr{L}(f(ct))(s) &= \int_0^\infty{e^{-st} f(ct) dt} \\
    &= \frac{-1}{s}e^{-st}f(ct) - \int_0^\infty{\frac{-1}{s}e^{-st}f'(ct) dt} \\
\end{align}

\ex{3}

\begin{enumerate}[label=\alph*)]
    \item 
        \begin{align}
            f(t) &= \sinh(t)\cos(t) \\
            F(t) &= \frac{1}{2}\int_0^\infty{e^{-st} (e^t - e^{-t})\cos(t) dt} \\
            &= \frac{1}{4}\int_0^\infty{e^{-st} (e^t - e^{-t})(e^{it} + e^{-it}) dt} \\
            &= \frac{1}{4}\int_0^\infty{e^{-st} (e^{it+t} - e^{it-t} +
                e^{-it+t} - e^{-it-t}) dt} \\
%% THIS PART WAS BEAUTIFUL, BUT IT WAS VERY VERY WRONG, WHAT A SHAME
%            &= \frac{1}{4}\int_0^\infty{e^{-st} (e^{it+t} - e^{-(it+t)}) +
%                e^{-st}(e^{-(it-t)} - e^{it-t}) dt} \\
%            &= \frac{1}{2}\int_0^\infty{e^{-st} \frac{1}{2}(e^{(it+t)} -
%                e^{-(it+t)}) dt} - \int_0^\infty{e^{-st}\frac{1}{2}(e^{(it-t)}
%                - e^{-(it-t)}) dt} \\
%            &= \frac{1}{2} \Big( \mathscr{L}\{\sinh(it+t)\} -
%                \mathscr{L}\{\sinh(it-t)\} \Big) \\
%            &= \frac{1}{2} \Bigg( \frac{it+t}{s^2 - (it+t)^2} -
%                \frac{it-t}{s^2 - (it-t)^2} \Bigg) \\
%            &= \frac{1}{2} \Bigg( \frac{it+t}{s^2 - 2it} -
%                \frac{it-t}{s^2 + 2it} \Bigg) \\
%            &= \frac{1}{2} \Bigg( \frac{(it+t)(s^2 + 2it)}{s^4 + 4t^2} -
%                \frac{(it-t)(s^2 - 2it)}{s^4 + 4t^2} \Bigg) \\
%            &= \frac{1}{2} \Bigg( \frac{(its^2 - 2t^2 + ts^2 + 2it^2) 
%            - (its^2 + 2t^2 - ts^2 + 2it^2)}{s^4 + 4t^2} \Bigg) \\
%            &= \frac{ts^2 - 2t^2}{s^4 + 4t^2}\\
            \intertext{This is the same as \(\frac{1}{2} \mathscr{L}
                \{\sinh(it+t)\} - \frac{1}{2} \mathscr{L} \{\sinh(it-t)\}\). By
                using this with the table for \(\frac{1}{a}\sinh{at}\) the
                final result was quite beautiful, but as I realise now it was
                also very very wrong (setting \(at = it\pm t\)). What a shame.}
            &= \frac{1}{4}\int_0^\infty{e^{it+t-st} - e^{it-t-st} +
                e^{-it+t-st} - e^{-it-t-st} dt} \\
            &= \frac{1}{4}\int_0^\infty{e^{t(i+1-s)} - e^{t(i-1-s)} +
                e^{t(-i+1-s)} - e^{t(-i-1-s)} dt} \\
            &= \frac{1}{4} \Big[ \frac{1}{i+1-s}e^{t(i+1-s)} -
                \frac{1}{i-1-s}e^{t(i-1-s)} + \frac{1}{-i+1-s}e^{t(-i+1-s)} -
                \frac{1}{-i-1-s}e^{t(-i-1-s)}  \Big]_0^\infty      \\
            &= \frac{1}{4} \Big[ \frac{e^{t(i-1-s)}}{(s+1)-i}
                +\frac{e^{t(-i-1-s)}}{(s+1)+i} - \frac{e^{t(-i+1-s)}}{(s-1)+i}
                - \frac{e^{t(i+1-s)}}{(s-1)-i} \Big]_0^\infty      \\
            &= \frac{1}{4} \Big[ \frac{e^{t(i-1-s)}((s+1)+i)
                + e^{t(-i-1-s)}((s+1)-i)}{(s+1)^2+1}
                - \frac{e^{t(-i+1-s)}((s-1)-i)
                + e^{t(i+1-s)}((s-1)+i)}{(s-1)^2+1} \Big]_0^\infty      \\
        \end{align}

    \item
        \begin{align}
            f(t) &= \cos^2(2t) \\
            \mathscr{L}\{f(t)\}(s) &= \mathscr{L}\{\frac{1}{2}+\frac{1}{2}\cos(4x)\} \\
            &= \frac{1}{2}\Big(\frac{1}{s}+\frac{s}{s^2+16}\Big) \\
            &= \frac{1}{2}\Big(\frac{s^2+16 + s^2}{s^3+16s}\Big) \\
            &= \frac{s^2+8}{s^3+16s} \\
        \end{align}
\end{enumerate}


\ex{4}

\begin{enumerate}[label=\alph*)]
    \item 
        \begin{align}
            F(s) &= \frac{4}{s^2-2s-3} \\
            &= \frac{4}{(s-3)(s+1)} = 4\mathscr{L}\{e^{3t}\}\mathscr{L}\{e^{-t}\} \\
            f(t) &= 4(e^{3t} * e^{-t})(t) \\
            &= 4\int_0^t{e^{3(t-\tau)}e^{-\tau}d\tau} \\
            &= 4\int_0^t{e^{3t-3\tau-\tau} d\tau} \\
            &= 4\int_0^t{e^{3t-4\tau} d\tau} \\
            &= -4\Big[\frac{1}{4}e^{3t-4\tau}\Big]_0^t\\
            &= e^{3t} - e^{-t}\\
        \end{align}

    \item
        \begin{align}
            F(s) &= \frac{4}{s^4-s^2} \\
            F(s) &= \frac{4}{s^2(s^2-1)} \\
            F(s) &= 4\frac{1}{s^2}\mathscr{L}\{\sinh(t)\} \\
            \intertext{We know that \(\mathscr{L}\{\int\int g(t)dtdt\} = \frac{1}{s^2}G(s)\) which means that}
            f(t) &= 4\int_0^t\int_0^t\sinh(t)dtdt \\
            &= \sinh(t) + c_1t + c_2\\
        \end{align}
\end{enumerate}


%% uncomment this if you need references. Edit the .bbl file with your references
%% and use "\cite{bibitem-label}" to cite
%\bibliography{template}

\end{document}

