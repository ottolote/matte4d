\documentclass[a4paper]{article}
\usepackage[english]{babel}
\usepackage[utf8]{inputenc}

% mathermatics
\usepackage{amssymb} % useful math symbols
\usepackage{amsmath} % more useful math
\usepackage{gensymb}

% references and automatic numbering
\usepackage{hyperref}
\usepackage{cleveref}
\usepackage{autonum}

% graphics
\usepackage{graphicx}
\usepackage{float}    % for more accurate graphics placement
\usepackage{fancyhdr} % for top header

% formatting
\usepackage{enumitem} % provides easy change of labels in enumerate environment
\usepackage[top=3cm, bottom=4cm, width=15cm]{geometry} % for smaller page margins

% colors
\usepackage{xcolor}

%% coding
% 
% uncomment this after you have completed the required installation
% see https://github.com/gpoore/minted for info
%
%\usepackage{minted} 
%\definecolor{codeBgColor}{RGB}{240,240,240}
%



% very simple alias, \ex{} becomes the same as \subsubsection*{}
% TIP: remove the * in the line below if you want it numbered
\newcommand{\ex}[1]{\subsubsection*{#1}}




%Begining of the document
\begin{document}

\pagestyle{fancy} % use pagestyle with simple header (from fancyhdr)

%\pagenumbering{gobble} % uncomment to remove pagenumbering (in case of single page document)
\fancyhead[L]{TMA4135 Matematikk 4D}
\fancyhead[C]{\textbf{Exercise 4}}
\fancyhead[R]{Otto Lote (748704)}


\ex{1}

\begin{align}
    \intertext{We have the heat equation}
    \frac{\partial u}{\partial t} &= c^2 \frac{\partial^2 u}{\partial x^2}
        \label{eq:1-def} \\
    \intertext{with initial condition}
    u(x,0) &= 200 \sin(\pi x) \label{eq:1-init} \\
    \intertext{The ends of the rod is kept at 0\degree C which means}
    u(0,t) &= 0 \quad\text{and}\quad u(L,t) = 0 \label{eq:1-constraints}\\
    \intertext{By variable separation we set}
    u(x,t) &= F(x)G(t) \label{eq:1-separated}\\
    \intertext{and}
    F(0) &= 0 \label{eq:1-F-constraints}\\ 
    F(L) &= 0 \label{eq:1-G-constraints}\\
    \intertext{which when inserted into (\ref{eq:1-def})gives us}
    F\dot G &= c^2 F''G \\
    \frac{F}{F''} &= c^2\frac{G}{\dot G} \\
    \intertext{Since F and G are independent both sides of this equation must
        be constant}
    \frac{F}{F''} &= c^2\frac{G}{\dot G} = k \\
    \intertext{and can be split into two ODEs}
    F -kF'' &= 0 \label{eq:1-F-ODE}\\
    c^2G - k\dot G &= 0 \label{eq:1-G-ODE}\\
    \intertext{To solve (\ref{eq:1-F-ODE}) we first try with positive \(k =
        -p^2\) which gives us}
    F(x) &= \frac{F'(0)}{p} \sin px \\
    \intertext{If we insert (\ref{eq:1-F-constraints}) into this we get}
    F(L) &= \frac{F'(0)}{p} \sin pL = 0 \\
    \intertext{giving us}
    p &= \frac{n\pi}{L} \\
    \intertext{and a general solution}
    F(x) &= \frac{F'(0)L}{n\pi}\sin \frac{n\pi}{L}x \\
\end{align}


\ex{2}
\begin{align} 
    \intertext{Since we have a steady state there is no change dependent on time, meaning all time derivatives are constant (or zero)}
    c^2 \frac{\partial^2 u}{\partial x^2} &= k \label{eq:2-def} \\
    \intertext{We use (\ref{eq:1-separated}) to get}
    c^2 F''(x)G(t) &= k \\
    \intertext{We know that G(t) is constant since we are in a steady state,
        which means}
    u(x,\infty) &= c^2G(\infty)F''(x) \\
    F''(x) &= K \\
    F(x) &= Kx^2 + \alpha x + \beta \\
    \intertext{We have that}
    F(0) &= U_1 \\
    F(L) &= U_2 \\
    \intertext{which gives us}
    \beta &= U_1
\end{align}


\ex{3}

%% uncomment this if you need references. Edit the .bbl file with your references
%% and use "\cite{bibitem-label}" to cite
%\bibliography{template}

\end{document}

