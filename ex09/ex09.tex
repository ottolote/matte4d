\documentclass[a4paper]{article}
\usepackage[english]{babel}
\usepackage[utf8]{inputenc}

% mathermatics
\usepackage{amssymb} % useful math symbols
\usepackage{mathrsfs,amsmath} % more useful math (mathrsfs for Fourier F)

% graphics
\usepackage{graphicx}
\usepackage{float}    % for more accurate graphics placement
\usepackage{fancyhdr} % for top header

% references
\usepackage{hyperref} % needed by cleveref, also provides clickable links
\usepackage{cleveref} % needed by autonum
\usepackage{autonum} % only add numbers to referenced equations

% formatting
\usepackage{enumitem} % provides easy change of labels in enumerate environment
\usepackage[top=3cm, bottom=4cm, width=17cm]{geometry} % for smaller page margins

% colors
\usepackage{xcolor}

% coding
% 
% uncomment this after you have completed the required installation
% see https://github.com/gpoore/minted for info

\usepackage{minted} 
\definecolor{codeBgColor}{RGB}{240,240,240}




% very simple alias, \ex{} becomes the same as \subsubsection*{}
% TIP: remove the * in the line below if you want it numbered
\newcommand{\ex}[1]{\subsubsection*{#1}}




%Begining of the document
\begin{document}

\pagestyle{fancy} % use pagestyle with simple header (from fancyhdr)

%\pagenumbering{gobble} % uncomment to remove pagenumbering (in case of single page document)
\fancyhead[L]{TMA4135 Matematikk 4D}
\fancyhead[C]{\textbf{Exercise 9}}
\fancyhead[R]{Otto Lote (748704)}
\fancyfoot{}

\ex{1}

\[f(x) = x^3 - x^2 + x + 2 = 0 \] 

\begin{enumerate}[label=\alph*)]
    \item
        \begin{align}
            \intertext{we differentiate \(f\) and get}
            f'(x) &= 3x^2 - 2x + 1\\
            \intertext{Since this is a polynomial with imaginary roots we know
                that \(f'\) is either larger than or less than zero. Since
                \(f(0) = 2\) we know that \(f\) is strictly ascending and
                one-to-one, and therefore has only one unique solution}
            \intertext{Finding and interval that contains the solution we see
                that \(f(0) = 2\) is close to zero. As \(f\) is ascending we
                try \(f(-1)\) and \(f(-2)\) and get}
            f(-1) &= -1 - (-1) + (-1) + 2 = 1 \\
            f(-2) &= -8 - 4 + (-2) + 2 = -12 \\
            \intertext{Since our solution \(s\) lies between \(f(-2) < f(s) <
                f(-1)\) we know that our solution lies on the interval \([-2,-1]\)}
            \intertext{}
        \end{align}

    \item
        \begin{align}
            \intertext{Since Newton's method is of second order its error is given by}
            \epsilon_{n+1} &\approx -\frac{f''(s)}{2f'(s)}\epsilon_n^2 \label{eq:1b-eps}\\
            \intertext{This means we need to complete one iteration of Newton's
            method to estimate the number of iterations required}
            x_{1} &= x_0 - \frac{f(x_0)}{f'(x_0)} = -1 - \frac{1}{6} = -\frac{7}{6} \\
            \intertext{The fraction in the error estimate is}
            \frac{f''(s)}{2f'(s)} &\approx \frac{f''(x_1)}{2f'(x_1)} 
                = \frac{6x_1 - 2}{6x_1^2 - 4x_1 + 2} = \frac{-7-2}{7^2 + 4.67 + 2} 
                = -0.16  \\
            \intertext{From (\ref{eq:1b-eps}) we have}
            |\epsilon_{n+1}| &\approx 0.16\epsilon_n^2 \approx 0.16(0.16\epsilon^2_{n-1})^2
                = 0.16^3\epsilon^4_{n-1} \approx 0.16^M\epsilon_0^{M+1} \leq 5
                \cdot 10^{-6}
            \intertext{Where \(M = 2^n + 2^{n-1} + ... + 2 + 1 = 2^{n+1} - 1\)}
            \intertext{To find \(\epsilon_0\) we have that}
            \epsilon_1 - \epsilon_0 &= (\epsilon_1 - s) - (\epsilon_0 - s) =
                -x_1 + x_0 \approx 0.17 \\
            \epsilon_1 &= \epsilon_0 + 0.17 \approx 0.16\epsilon_0^2 +
                \epsilon_0 + 0.17 \approx 0 \\
            \intertext{This gives us \(\epsilon_0 \approx -0.175\) which gives
                us the inequality}
            0.16^M0.175^{M+1} &\leq 5 \cdot 10^{-6} \\
            \intertext{This holds for \(n=1\) which means Newton's method has
                (more than) 5 decimals of accuracy after 1 iteration}
        \end{align}
    \item Since 1 iteration was enough we have already approximated the
        solution to 5 decimals of precision with \(x_1 = -\frac{7}{6}\)
\end{enumerate}


\ex{2}

\begin{align}
\end{align}

\ex{3}

\begin{enumerate}[label=\alph*)]
    \item 
    \item 
        \begin{align}
        \end{align}

    \item 
        \begin{align}
        \end{align}
    \item 
        \begin{align}
        \end{align}
\end{enumerate}

\ex{4}

\begin{align}
\end{align}



%% uncomment this if you need references. Edit the .bbl file with your references
%% and use "\cite{bibitem-label}" to cite
%\bibliography{template}

\end{document}

