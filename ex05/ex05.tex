\documentclass[a4paper]{article}
\usepackage[english]{babel}
\usepackage[utf8]{inputenc}

% mathermatics
\usepackage{amssymb} % useful math symbols
\usepackage{amsmath} % more useful math
\usepackage{gensymb}

% references and automatic numbering
\usepackage{hyperref}
\usepackage{cleveref}
\usepackage{autonum}

% graphics
\usepackage{graphicx}
\usepackage{float}    % for more accurate graphics placement
\usepackage{fancyhdr} % for top header

% formatting
\usepackage{enumitem} % provides easy change of labels in enumerate environment
\usepackage[top=3cm, bottom=4cm, width=15cm]{geometry} % for smaller page margins

% colors
\usepackage{xcolor}

%% coding
% 
% uncomment this after you have completed the required installation
% see https://github.com/gpoore/minted for info
%
%\usepackage{minted} 
%\definecolor{codeBgColor}{RGB}{240,240,240}
%



% very simple alias, \ex{} becomes the same as \subsubsection*{}
% TIP: remove the * in the line below if you want it numbered
\newcommand{\ex}[1]{\subsubsection*{#1}}




%Begining of the document
\begin{document}

\pagestyle{fancy} % use pagestyle with simple header (from fancyhdr)

%\pagenumbering{gobble} % uncomment to remove pagenumbering (in case of single page document)
\fancyhead[L]{TMA4135 Matematikk 4D}
\fancyhead[C]{\textbf{Exercise 5}}
\fancyhead[R]{Otto Lote (748704)}


The fourier transform is given by
\[\hat f(x) = \frac{1}{\sqrt{2\pi}}\int_{-\infty}^{\infty}{f(x)e^{-iwx} dx}
    \label{ftrans}\]


\ex{1}
\[ f(x) = e^-|x| = 
    \begin{cases}
        f(x) = e^-x, &\quad x >= 0 \\
        f(x) = e^x, &\quad x < 0 
    \end{cases} \label{eq:1-fdef} \]

\begin{enumerate}[label=\alph*)]
    \item { 
        \begin{align}
            \intertext{Inserting (\ref{eq:1-fdef}) into (\ref{ftrans}) we have that}
            \hat f(x) &= \frac{1}{\sqrt{2\pi}}\Bigg(\int_{-\infty}^0{e^x
                e^{-iwx} dx} + \int_0^{\infty}{e^{-x} e^{-iwx} dx}\Bigg) \\
            \hat f(x) &= \frac{1}{\sqrt{2\pi}}\Bigg(\int_{-\infty}^0{e^{x-iwx}dx} 
                + \int_0^{\infty}{e^{-iwx-x} dx}\Bigg) \\
            &= \frac{1}{\sqrt{2\pi}}\Bigg(\Big[{\frac{1}{1-iw}e^{(1-iw)x}\Big]_
                {-\infty}^{0}} - \Big[{\frac{1}{1+iw}
                e^{-(1+iw)x}\Big]_{0}^{\infty}}\Bigg) \\
            &= \frac{1}{\sqrt{2\pi}}\Bigg(\frac{1}{1-iw} + \frac{1}{1+iw}\Bigg) \\
            &= \frac{1}{\sqrt{2\pi}}\Bigg(\frac{1+iw}{1+w^2} +
                \frac{1-iw}{1+w^2}\Bigg) = \frac{1}{\sqrt{2\pi}(1 + w^2)}\\
        \end{align}
    }

    \item{  % wat
        \begin{align}
            \int_0^\infty{\frac{1}{1+x^2} dx} &= \int_0^\infty{\frac{1}{1+ix} +
                \frac{1}{1-ix} dx} = \Big[ \frac{1}{i}\ln|1+ix| -
                \frac{1}{i}\ln|1-ix|\Big]_0^\infty\\
            &= \Big[ i\Big(\ln|1-ix| - \ln|1+ix|\Big)\Big]_0^\infty \\
            &= \Big[ i\ln\Big(\frac{|1-ix|}{|1+ix|}\Big)\Big]_0^\infty \\
        \end{align}
    }
\end{enumerate}


\end{document}

