\documentclass[a4paper]{article}
\usepackage[english]{babel}
\usepackage[utf8]{inputenc}

% mathermatics
\usepackage{amssymb} % useful math symbols
\usepackage{amsmath} % more useful math
\usepackage{gensymb}

% references and automatic numbering
\usepackage{hyperref}
\usepackage{cleveref}
\usepackage{autonum}

% graphics
\usepackage{graphicx}
\usepackage{float}    % for more accurate graphics placement
\usepackage{fancyhdr} % for top header

% formatting
\usepackage{enumitem} % provides easy change of labels in enumerate environment
\usepackage[top=3cm, bottom=4cm, width=15cm]{geometry} % for smaller page margins

% colors
\usepackage{xcolor}

%% coding
% 
% uncomment this after you have completed the required installation
% see https://github.com/gpoore/minted for info
%
%\usepackage{minted} 
%\definecolor{codeBgColor}{RGB}{240,240,240}
%



% very simple alias, \ex{} becomes the same as \subsubsection*{}
% TIP: remove the * in the line below if you want it numbered
\newcommand{\ex}[1]{\subsubsection*{#1}}




%Begining of the document
\begin{document}

\pagestyle{fancy} % use pagestyle with simple header (from fancyhdr)

%\pagenumbering{gobble} % uncomment to remove pagenumbering (in case of single page document)
\fancyhead[L]{TMA4135 Matematikk 4D}
\fancyhead[C]{\textbf{Exercise 5}}
\fancyhead[R]{Otto Lote (748704)}


The fourier transform is given by
\[\hat f(x) = \frac{1}{\sqrt{2\pi}}\int_{-\infty}^{\infty}{f(x)e^{-i\omega x} dx}
    \label{ftrans}\]


\ex{1}
\[ f(x) = e^-|x| = 
    \begin{cases}
        f(x) = e^-x, &\quad x >= 0 \\
        f(x) = e^x, &\quad x < 0 
    \end{cases} \label{eq:1-fdef} \]

\begin{enumerate}[label=\alph*)]
    \item { 
        \begin{align}
            \intertext{Inserting (\ref{eq:1-fdef}) into (\ref{ftrans}) we have that}
            \hat f(\omega ) &= \frac{1}{\sqrt{2\pi}}\Bigg(\int_{-\infty}^0{e^x
                e^{-i\omega x} dx} + \int_0^{\infty}{e^{-x} e^{-i\omega x} dx}\Bigg) \\
            \hat f(\omega ) &= \frac{1}{\sqrt{2\pi}}\Bigg(\int_{-\infty}^0{e^{x-i\omega x}dx} 
                + \int_0^{\infty}{e^{-i\omega x-x} dx}\Bigg) \\
            &= \frac{1}{\sqrt{2\pi}}\Bigg(\Big[{\frac{1}{1-i\omega }e^{(1-i\omega )x}\Big]_
                {-\infty}^{0}} - \Big[{\frac{1}{1+i\omega }
                e^{-(1+i\omega )x}\Big]_{0}^{\infty}}\Bigg) \\
            &= \frac{1}{\sqrt{2\pi}}\Bigg(\frac{1}{1-i\omega } + \frac{1}{1+i\omega }\Bigg) \\
            &= \frac{1}{\sqrt{2\pi}}\Bigg(\frac{1+i\omega }{1+\omega ^2} +
                \frac{1-i\omega }{1+\omega ^2}\Bigg) = \frac{1}{\sqrt{2\pi}(1 + \omega ^2)}\\
        \end{align}
    }

    \item{  % wat
        \begin{align}
            \int_0^\infty{\frac{1}{1+x^2} dx} &= \Big[\arctan x\Big]_0^\infty =
                \arctan \infty - \arctan 0 = \frac{\pi}{2}
        \end{align}
    }
\end{enumerate}

\ex{2}

\begin{align}
f(x) = 
    \begin{cases}
        \sin(x) &\quad -\pi < x < \pi \\
        0 &\quad \text{otherwise} \\
    \end{cases}
\end{align}

\begin{enumerate}[label=\alph*)]
    \item {
        \begin{align}
            \hat f(\omega ) &= \frac{1}{\sqrt{2\pi}} \int_{-\infty}^\infty{ 
                \sin(x) e^{-i\omega x} dx} \\
            &= \frac{1}{\sqrt{2\pi}} \int_{-\pi}^\pi{\sin(x) e^{-i\omega x} dx} \\
            \intertext{We use the Euler identity \(\sin(x) = \frac{1}{2i}(e^{ix}
                - e^{-ix})\)}
            \hat f(\omega ) &= \frac{1}{\sqrt{2\pi}} \int_{-\pi}^\pi{ \frac{1}{2i}(e^{ix}
                - e^{-ix}) e^{-i\omega x} dx} \\
            &= \frac{1}{\sqrt{2\pi}} \int_{-\pi}^\pi{ \frac{1}{2i}\big( e^{ix}e^{-i\omega x}
                - e^{-ix} e^{-i\omega x}\big) dx} \\
            &= \frac{1}{\sqrt{2\pi}} \int_{-\pi}^\pi{ \frac{1}{2i} e^{(i-i\omega )x}
                - \frac{1}{2i} e^{-(i+i\omega )x} dx} \\
            &= \frac{1}{\sqrt{2\pi}} \Big[\frac{1}{2i(i-i\omega )} e^{(i-i\omega )x} 
                + \frac{1}{2i(i+i\omega )} e^{-(i+i\omega )x} \Big]_{-\pi}^\pi \\
            &= \frac{1}{\sqrt{2\pi}} \Big[\frac{-1}{2(1-\omega )} e^{(i-i\omega )x} 
                - \frac{1}{2(1+\omega )} e^{-(i+i\omega )x} \Big]_{-\pi}^\pi \\
            &= \frac{1}{\sqrt{2\pi}} \Big(\frac{-1}{2(1-\omega )} e^{(i-i\omega )\pi} 
                - \frac{1}{2(1+\omega )} e^{-(i+i\omega )\pi} \Big)
                - \frac{1}{\sqrt{2\pi}} \Big(\frac{-1}{2(1-\omega )} e^{-(i-i\omega )\pi} 
                - \frac{1}{2(1+\omega )} e^{(i+i\omega )\pi} \Big) \\
            &= \frac{1}{\sqrt{2\pi}} \Bigg( 
                - \frac{1}{2(1-\omega )} e^{(i-i\omega )\pi} 
                - \frac{1}{2(1+\omega )} e^{-(i+i\omega )\pi} 
                + \frac{1}{2(1-\omega )} e^{-(i-i\omega )\pi} 
                + \frac{1}{2(1+\omega )} e^{(i+i\omega )\pi} \Bigg) \\
            &= \frac{1}{2\sqrt{2\pi}} \Bigg( 
                - \frac{1}{(1-\omega )} e^{i\pi}e^{-i\omega \pi} 
                - \frac{1}{(1+\omega )} e^{-i\pi}e^{-i\omega \pi} 
                + \frac{1}{(1-\omega )} e^{-i\pi}e^{i\omega \pi} 
                + \frac{1}{(1+\omega )} e^{i\pi}e^{i\omega \pi} \Bigg) \\
            &= \frac{1}{2\sqrt{2\pi}} \Bigg( 
                - \frac{1}{(1-\omega )} e^{i\pi}e^{-i\omega \pi} 
                - \frac{1}{(1+\omega )} e^{-i\pi}e^{-i\omega \pi} 
                + \frac{1}{(1-\omega )} e^{-i\pi}e^{i\omega \pi} 
                + \frac{1}{(1+\omega )} e^{i\pi}e^{i\omega \pi} \Bigg) \\
        \end{align}
    }
\end{enumerate}

\ex{3}

\begin{align}
    \intertext{We know  that \( \widehat{f'}(f'(x)) = i\omega \hat f(\omega )\) and therefore 
        \(\widehat{f''}(f''(x)) = \omega ^2 \hat f(\omega )\)}
    \intertext{we also know that for \(f(x) = \frac{1}{4}e^{-x^2}\), \(f''(x) = x^2e^{-x^2}\)}
    \hat f(\omega ) &= \frac{1}{4\sqrt{2\pi}}\int_{-\infty}^{\infty}{e^{-x^2} e^{-i\omega x} dx} \\
    &= \frac{1}{4\sqrt{2\pi}}\int_{-\infty}^{\infty}{e^{-x^2-i\omega x} dx} \\
    &= \frac{1}{4\sqrt{2\pi}}\Big(\int_{-\infty}^{0}{e^{-x^2-i\omega x} dx}
        + \int_{0}^{\infty}{e^{-x^2-i\omega x} dx}\Big) \\
    &= \frac{1}{4\sqrt{2\pi}}\Big(\Big[{\frac{-1}{2x+1}e^{-x^2-i\omega x}\Big]_{-\infty}^{0}} 
        + \Big[{\frac{-1}{2x+1}e^{-x^2-i\omega x}\Big]_{0}^{\infty}}\Big) \\
    &= \frac{1}{4\sqrt{2\pi}}\Bigg(\Big( (-1) - 0\Big) + \Big(0 - 1 \Big)\Bigg) \\
\end{align}


\ex{4}

\begin{align}
    f(x) &= e^{-x^2} \\
    g(x) &= xe^{-x^2} \\
    f'(x) &= -2g(x) \\
    \hat f(w) &= \frac{1}{\sqrt{2}} e^{-\frac{\omega^2}{4}} \\
    \hat g(x) &= -\frac{1}{2} \widehat{f'}(\omega) = -\frac{1}{2}i\omega \hat f(\omega) \\
    \hat f(\omega) \hat g(\omega) &= -\frac{1}{2}i\omega \hat f(\omega)^2 \\
    &= -i\omega\frac{1}{4}e^{-\frac{\omega^2}{2}} \\
    (f*g)(x) &= \mathcal{F}^{-1} ( \hat f(\omega) \hat g(\omega) ) \\
    &= -\frac{i}{4} \int_{-\infty}^{\infty}{\omega e^{-\frac{\omega^2}{2}}
        e^{i\omega x} d\omega}
\end{align}


\end{document}

