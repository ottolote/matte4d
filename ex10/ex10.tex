\documentclass[a4paper]{article}
\usepackage[english]{babel}
\usepackage[utf8]{inputenc}

% mathermatics
\usepackage{amssymb} % useful math symbols
\usepackage{mathrsfs,amsmath} % more useful math (mathrsfs for Fourier F)

% graphics
\usepackage{graphicx}
\usepackage{float}    % for more accurate graphics placement
\usepackage{fancyhdr} % for top header

% references
\usepackage{hyperref} % needed by cleveref, also provides clickable links
\usepackage{cleveref} % needed by autonum
\usepackage{autonum} % only add numbers to referenced equations

% formatting
\usepackage{enumitem} % provides easy change of labels in enumerate environment
\usepackage[top=3cm, bottom=4cm, width=17cm]{geometry} % for smaller page margins
\usepackage{multirow}

% colors
\usepackage{xcolor}

% coding
% 
% uncomment this after you have completed the required installation
% see https://github.com/gpoore/minted for info

\usepackage{minted} 
\definecolor{codeBgColor}{RGB}{240,240,240}




% very simple alias, \ex{} becomes the same as \subsubsection*{}
% TIP: remove the * in the line below if you want it numbered
\newcommand{\ex}[1]{\subsubsection*{#1}}




%Begining of the document
\begin{document}

\pagestyle{fancy} % use pagestyle with simple header (from fancyhdr)

%\pagenumbering{gobble} % uncomment to remove pagenumbering (in case of single page document)
\fancyhead[L]{TMA4135 Matematikk 4D}
\fancyhead[C]{\textbf{Exercise 10}}
\fancyhead[R]{Otto Lote (748704)}
\fancyfoot{}

\ex{1}

We have the data points 

\begin{tabular}{c | r r r r r}
    \(x_i\) & 0 & 1 & 2 & 4 & 6 \\
    \hline
    \(f(x_i)\) & 4 & -1 & -3 & -6 & -9 \\
\end{tabular}

\begin{enumerate}[label=\alph*)]
    \item
        We start by arranging the divided difference table

        \bgroup
        \def\arraystretch{1.8}%
        \begin{tabular}{c r | r r r r}
            \(x_j\) & \(f_j = f(x_j)\) & \(f[x_j, x_{j+1}]\) & \(f[x_j,
                x_{j+1}, x_{j+2}]\) & \(f[x_j, ..., x_{j+3}]\) &
                \(f[x_j, ...,  x_{j+4}]\) \\
            \hline
            0 & \color{red}{4} & \multirow{2}{*}{\color{red}{-5}} 
                & \multirow{3}{*}{\color{red}{1.5}} 
                & \multirow{4}{*}{\color{red}{-0.333333}} 
                & \multirow{5}{*}{\color{red}{0.061111}} \\
            1 & -1 & \multirow{2}{*}{-2} & \multirow{3}{*}{0.166667} 
                & \multirow{4}{*}{0.033333} \\
            2 & -3 & \multirow{2}{*}{-1.5} & \multirow{3}{*}{0} \\
            4 & -6 & \multirow{2}{*}{-1.5} \\
            6 & -9 \\
        \end{tabular}
        \egroup

        \begin{align}
            \intertext{Using the values marked red in the table above we have}
            f(x) &\approx p_4(x) = 4 - 5x + 1.5x(x-1) - 0.333333x(x-1)(x-2) +
                0.061111x(x-1)(x-2)(x-4) \\
        \end{align}

    \item
        \begin{align}
            \intertext{For \(x=5\) we get}
            f(5) &\approx 4 - 25 + 30 - 20 + 3.66 = -7.333333 \\
        \end{align}

    %\item .
\end{enumerate}

\newpage
\ex{2}

\[f(x) = x^2 \cos x\]

\begin{enumerate}[label=\alph*)]
    \item 
        \begin{align}
            \intertext{We are after a polynomial of degree 3. Since a
                polynomial interpolation with \(n\) nodes produce a polynomial
                of degree \(n-1\) at most we will use 4 Chebyshev nodes for
                approximation, given by}
            x_k &= \frac{1}{2}(a+b) - \frac{1}{2}(b-a) \cos\Big(
                \frac{2k-1}{2n} \pi \Big), \quad k=1, ..., n \\ 
            x_k &= \frac{1}{2} - \frac{3}{2} \cos \Big( \frac{2k-1}{2n} \Big)
                \pi, \quad k = 1, 2, 3, 4 \\
        \end{align}

        Four Chebyshev nodes on the interval \([-1, 2]\) are given by

        \begin{tabular}{c | c}
            \(k\) & \(x_k\) \\
            \hline
            1 & -0.885819299 \\
            2 & -0.074025149 \\
            3 & 1.074025149 \\
            4 & 1.885819299 \\
        \end{tabular}

        \begin{align}
            \intertext{We have the general Langrance Interpolation Polynomial}
            f(x) &\approx p_n(x) = \sum_{k=0}^n{L_n(x)f_k} =
                \sum_{k=0}^n{\frac{l_k(x)}{l_k(x_k)}f_k} \\
            \intertext{where}
            l_0(x) &= (x-x_1)(x-x_2)...(x-x_n), \\
            l_k(x) &= (x-x_0)...(x-x_{k-1})(x-x_{k+1}...(x-x_n), \\
            l_n(x) &= (x-x_0)(x-x_1)...(x-x_{n-1}) \\
            \intertext{For our four Chebyshev nodes we get}
            l_0(x) &= (x+0.074025149)(x-1.074025149)(x-1.885819299), \\
            l_1(x) &= (x+0.885819299)(x-1.074025149)(x-1.885819299), \\
            l_2(x) &= (x+0.885819299)(x+0.074025149)(x-1.885819299), \\
            l_3(x) &= (x+0.885819299)(x+0.074025149)(x-1.074025149), \\
            p_n(x) &= 
                \frac{(x+0.074025149)(x-1.074025149)(x-1.885819299)}{l_0(x_0)}  \\
                &+ \frac{(x+0.885819299)(x-1.074025149)(x-1.885819299)}{l_1(x_1)} \\
                &+ \frac{(x+0.885819299)(x+0.074025149)(x-1.885819299)}{l_2(x_2)} \\
                &+ \frac{(x+0.885819299)(x+0.074025149)(x-1.074025149)}{l_3(x_3)}  
        \end{align}

    \item 
        \begin{align}
        \end{align}
\end{enumerate}

\ex{3}

\[\int_1^3 e^{-x} dx\]
\begin{enumerate}[label=\alph*)]
    \item 
        \begin{align}
            \intertext{The error \(\epsilon_S\) of Simpson's Rule is given by}
            \epsilon_S &= - \frac{(b-a)^5}{180(2m)^4} f^{(4)}(\hat t) \\
            f^{(4)} &= e^{-x} \\
            \intertext{and has its largest value at \(x = 1, f^{(4)}(1) =
                \frac{1}{e}\) and its smallest at \(x = 3, f^{(4)}(1) = \frac{1}{e^3}\)}
            \intertext{We set \(M_4 = \frac{1}{e}\), with \(b-a = 2\) and the
                required accuracy}
            |CM_4| &= \frac{2^5}{e \cdot 180(2m)^4} = 0.5 \cdot 10^{-5} \\
            \intertext{This gives us \(m \approx 5.35\) which means we need \(n
                = 2m = 12\) nodes}
        \end{align}

    \item 
        \begin{align}
            \intertext{We change the integral to be from -1 to 1 by setting \(x
                = t + 2\), and \(dx = dt\)}
            \int_1^3 e^{-x} dx &= \int_{-1}^1 e^{-t-2} dt = \int_{-1}^1 e^{-2}
                e^{-t} dt \\
        \end{align}

        For Gauss integration with 4 nodes we have the following table

        \begin{tabular}{c | c | c | c }
            Nodes \(t_j\) & Coefficients \(A_j\) & \(f(t_j)\) & \(A_jf(t_j)\)\\
            \hline
            -0.8611363116 & 0.3478548451 & 0.3201826424 & 0.1113770835 \\
            -0.3399810436 & 0.6521451549 & 0.1901353758 & 0.1239958647 \\
            0.3399810436 & 0.6521451549 & 0.09632946427 & 0.0628207937 \\
            0.8611363116 & 0.3478548451 & 0.05720372207 & 0.0198985919 \\
            \hline
            \multicolumn{2}{}{} & \(\sum_1^4{A_jf(t_j)}:\) & 0.3180923337 \\
        \end{tabular}

        This is an error of 0.00000003908 (below) the actual value of
            \(\frac{e^2 - 1}{e^3}\)
\end{enumerate}



%% uncomment this if you need references. Edit the .bbl file with your references
%% and use "\cite{bibitem-label}" to cite
%\bibliography{template}

\end{document}

